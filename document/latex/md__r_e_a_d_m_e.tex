Repositorio inicial para el Trabajo Practico 1 de la asignatura de Ingenieria de Software

\subsection*{Primera Parte}


\begin{DoxyEnumerate}
\item Clonar este repositorio en su computadora y desplegar la rama {\itshape master}.
\item Agregar al archivo alumnos una funcion que serialice sus datos personales similar a la del siguiente ejemplo. La declaración y la definición de la funcion debe agregarse abajo de las existentes.

```c bool Esteban\+Volentini(char $\ast$ cadena, size\+\_\+t espacio) \{ struct \hyperlink{structalumno__s}{alumno\+\_\+s} alumno = \{ .apellidos = \char`\"{}\+V\+O\+L\+E\+N\+T\+I\+N\+I\char`\"{}, .nombres = \char`\"{}\+Esteban Daniel\char`\"{}, .documento = \char`\"{}23.\+517.\+968\char`\"{}, \};

Serializar\+Alumno(cadena, sizeof(cadena), \&alumno); \} ```
\item Confirmar los cambios, resolver los conflictos y subir los cambios al servidor.
\end{DoxyEnumerate}

\subsection*{Segunda Parte}


\begin{DoxyEnumerate}
\item Crear un repositorio personal realizando un {\itshape fork} del repositorio de la catedra, clonar el repositorio personal en su computadora y desplegar la rama {\itshape develop}.
\item Editar la definicion de la constante A\+L\+U\+M\+N\+OS siguiendo el siguiente ejemplo

```c static const struct \hyperlink{structalumno__s}{alumno\+\_\+s} E\+S\+T\+E\+B\+A\+N\+\_\+\+V\+O\+L\+E\+N\+T\+I\+NI = \{ .apellidos = \char`\"{}\+V\+O\+L\+E\+N\+T\+I\+N\+I\char`\"{}, .nombres = \char`\"{}\+Esteban Daniel\char`\"{}, .documento = \char`\"{}23.\+517.\+968\char`\"{}, \};

const alumno\+\_\+t A\+L\+U\+M\+N\+OS\mbox{[}\mbox{]} = \{ \&E\+S\+T\+E\+B\+A\+N\+\_\+\+V\+O\+L\+E\+N\+T\+I\+NI, \}; ```
\item Confirmar los cambios, subirlos al servidor y pedir un {\itshape pull request} poniendo en la descripcion \char`\"{}\+Agrego los datos del alumno A\+P\+E\+L\+L\+I\+D\+O, Nombre\char`\"{}
\end{DoxyEnumerate}

\subsection*{Tercera Parte}


\begin{DoxyEnumerate}
\item Documentar los archivos {\ttfamily \hyperlink{alumnos_8h}{alumnos.\+h}} y {\ttfamily \hyperlink{alumnos_8c}{alumnos.\+c}} siguiendo los criterios proporcionados en la clase practica.
\item Modificar el archivo {\ttfamily makefile} para agregar una regla que genere la documentacion con el comando {\ttfamily make doc}
\item Confirmar los cambios y subirlos al servidor. 
\end{DoxyEnumerate}